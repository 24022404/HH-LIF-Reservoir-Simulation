\documentclass[12pt,a4paper]{article}

% Packages
\usepackage[utf8]{inputenc}
\usepackage[T5]{fontenc}
\usepackage[vietnamese]{babel}
\usepackage[margin=2.5cm]{geometry}
\usepackage{amsmath,amssymb}
\usepackage{graphicx}
\usepackage{hyperref}
\usepackage{listings}
\usepackage{xcolor}
\usepackage{booktabs}
\usepackage{float}
\usepackage{subcaption}

% Code listing settings
\lstset{
    basicstyle=\ttfamily\footnotesize,
    keywordstyle=\color{blue},
    commentstyle=\color{gray},
    numbers=left,
    numberstyle=\tiny\color{gray},
    frame=single,
    breaklines=true,
    xleftmargin=1.5em
}

\begin{document}

% Title page
\begin{titlepage}
    \centering
    
    % Frame border - even larger size
    \begin{minipage}[c][0.97\textheight]{0.96\textwidth}
        \centering
        \fbox{\begin{minipage}[c][0.95\textheight]{0.98\textwidth}
            \centering
            \vspace{1.5cm}
            
            % University name
            {\large\bfseries ĐẠI HỌC QUỐC GIA HÀ NỘI\\
            TRƯỜNG ĐẠI HỌC CÔNG NGHỆ\\[1.5cm]}
            
            % Logo - Place UET_logo.png in the same folder
            \includegraphics[width=4.5cm]{UET_logo.png}\\[1.5cm]
            
            % Report title
            {\Large\bfseries BÁO CÁO CÁ NHÂN\\[0.5cm]}
            {\large\bfseries KHÁM PHÁ CÁC MÔ HÌNH NEURON VÀ\\[0.2cm]
            MẠNG NEURAL\\[2.5cm]}
            
            % Team/Student info
            {\large\textbf{Sinh viên thực hiện:}\\[0.8cm]}
            \begin{flushleft}
            \hspace{3.5cm}1. \hspace{0.5cm} [Họ và tên] - [MSSV]\\[0.5cm]
            \end{flushleft}
            
            \vfill
            
            % Date
            {\large Hà Nội -- 2025}
            
            \vspace{1.5cm}
        \end{minipage}}
    \end{minipage}
    
\end{titlepage}

\tableofcontents
\newpage

% ============================================================
\section{Giới thiệu}

Báo cáo này trình bày kết quả mô phỏng ba mô hình tính toán thần kinh quan trọng:

\begin{enumerate}
    \item \textbf{Hodgkin-Huxley Model}: Mô phỏng điện thế hoạt động (action potential) với phân tích vai trò của dòng ion Na$^+$ và K$^+$
    \item \textbf{Leaky Integrate-and-Fire Model}: Mô phỏng neuron nhận dòng điện dạng sóng vuông
    \item \textbf{Reservoir Neural Network}: Dự đoán chuỗi thời gian Mackey-Glass ở khoảng cách gần ($t+10$) và xa ($t+100$)
\end{enumerate}

Các mô phỏng được thực hiện bằng Python với thư viện NumPy, SciPy, và matplotlib.

% ============================================================
\section{Task 1: Hodgkin-Huxley Model}

\subsection{Yêu cầu}
\begin{itemize}
    \item Định nghĩa tham số và 6 hàm tốc độ ($\alpha_x$, $\beta_x$)
    \item Điện thế ban đầu $V_0 = -65$ mV
    \item Dòng kích thích $I_{\text{ext}} = 20$ $\mu$A/cm$^2$ (step input)
    \item Vẽ đồ thị: $V$, $I_{\text{Na}}$, $I_{\text{K}}$
    \item Giải thích vai trò của $I_{\text{Na}}$ và $I_{\text{K}}$
\end{itemize}

\subsection{Tham số mô hình}

\begin{table}[H]
\centering
\caption{Tham số Hodgkin-Huxley}
\begin{tabular}{lcc|lcc}
\toprule
\textbf{Tham số} & \textbf{Giá trị} & \textbf{Đơn vị} & \textbf{Tham số} & \textbf{Giá trị} & \textbf{Đơn vị} \\
\midrule
$C_m$ & 1.0 & $\mu$F/cm$^2$ & $E_{\text{Na}}$ & 50.0 & mV \\
$g_{\text{Na}}$ & 120.0 & mS/cm$^2$ & $E_{\text{K}}$ & -77.0 & mV \\
$g_{\text{K}}$ & 36.0 & mS/cm$^2$ & $E_{\text{L}}$ & -54.387 & mV \\
$g_{\text{L}}$ & 0.3 & mS/cm$^2$ & $V_0$ & -65.0 & mV \\
\bottomrule
\end{tabular}
\end{table}

\subsection{Phương trình mô hình}

Hệ phương trình Hodgkin-Huxley:

\begin{align}
C_m \frac{dV}{dt} &= I_{\text{ext}} - I_{\text{Na}} - I_{\text{K}} - I_{\text{L}} \\
I_{\text{Na}} &= g_{\text{Na}} m^3 h (V - E_{\text{Na}}) \\
I_{\text{K}} &= g_{\text{K}} n^4 (V - E_{\text{K}}) \\
I_{\text{L}} &= g_{\text{L}} (V - E_{\text{L}})
\end{align}

\textbf{6 hàm tốc độ:}

\begin{align}
\alpha_m(V) &= \frac{0.1(V+40)}{1-e^{-(V+40)/10}}, \quad 
\beta_m(V) = 4e^{-(V+65)/18} \\
\alpha_h(V) &= 0.07e^{-(V+65)/20}, \quad 
\beta_h(V) = \frac{1}{1+e^{-(V+35)/10}} \\
\alpha_n(V) &= \frac{0.01(V+55)}{1-e^{-(V+55)/10}}, \quad 
\beta_n(V) = 0.125e^{-(V+65)/80}
\end{align}

Biến trạng thái cổng ion:
\begin{equation}
\frac{dx}{dt} = \alpha_x(V)(1-x) - \beta_x(V)x, \quad x \in \{m, h, n\}
\end{equation}

\subsection{Kết quả mô phỏng}

\begin{figure}[H]
\centering
\includegraphics[width=0.95\textwidth]{hh_static_plots.png}
\caption{Mô phỏng Hodgkin-Huxley: (a) Điện thế màng, (b) Dòng Sodium, (c) Dòng Potassium. Neuron phát xung lặp lại với tần số ~10 Hz khi nhận dòng kích thích 20 $\mu$A/cm$^2$.}
\end{figure}

\begin{figure}[H]
\centering
\includegraphics[width=0.75\textwidth]{/mnt/project/hh_action_potential.gif}
\caption{Animation quá trình hình thành action potential theo thời gian thực.}
\end{figure}

\subsection{Phân tích vai trò của $I_{\text{Na}}$ và $I_{\text{K}}$}

\subsubsection{Depolarization (Khử cực) - vai trò của $I_{\text{Na}}$}

\textbf{Thời gian:} $t \approx 5-8$ ms

\textbf{Cơ chế:}
\begin{itemize}
    \item Khi dòng kích thích được áp vào, $V$ tăng nhẹ
    \item Khi $V$ vượt ngưỡng $\approx -55$ mV, kênh Na$^+$ mở nhanh (biến $m$ tăng)
    \item Na$^+$ lao vào tế bào $\Rightarrow$ $I_{\text{Na}}$ âm (dòng vào)
    \item \textbf{Phản hồi dương:} $V \uparrow \Rightarrow$ kênh Na mở $\Rightarrow I_{\text{Na}} \downarrow \Rightarrow V \uparrow$
    \item Điện thế tăng nhanh: $-65$ mV $\rightarrow$ $+41$ mV trong $\sim$2 ms
\end{itemize}

\textbf{Kết quả quan sát:}
\begin{itemize}
    \item $V_{\text{peak}} = 41.29$ mV tại $t = 6.51$ ms
    \item $I_{\text{Na,min}} = -797.44$ $\mu$A/cm$^2$ (cực đại âm)
\end{itemize}

\subsubsection{Repolarization (Tái phân cực) - vai trò của $I_{\text{K}}$}

\textbf{Thời gian:} $t \approx 8-15$ ms

\textbf{Cơ chế:}
\begin{itemize}
    \item Kênh K$^+$ mở chậm hơn kênh Na (biến $n$ tăng dần)
    \item Kênh Na bắt đầu đóng (inactivation qua biến $h$)
    \item K$^+$ chảy ra ngoài $\Rightarrow$ $I_{\text{K}}$ dương (dòng ra)
    \item Điện thế giảm về giá trị nghỉ
    \item Kênh K đóng chậm $\Rightarrow$ hyperpolarization tạm thời
\end{itemize}

\textbf{Kết quả quan sát:}
\begin{itemize}
    \item $I_{\text{K,max}} = 850.28$ $\mu$A/cm$^2$ tại $t = 7.40$ ms
    \item Điện thế trở về $-65$ mV sau $\sim$10 ms
\end{itemize}

\subsubsection{Kết luận}

\begin{itemize}
    \item \textbf{$I_{\text{Na}}$}: Nguyên nhân chính của \textbf{depolarization}
    \begin{itemize}
        \item Mở nhanh, đóng nhanh
        \item Tạo pha tăng nhanh của action potential
        \item Cơ chế phản hồi dương khuếch đại tín hiệu
    \end{itemize}
    
    \item \textbf{$I_{\text{K}}$}: Nguyên nhân chính của \textbf{repolarization}
    \begin{itemize}
        \item Mở chậm, đóng chậm
        \item Đưa điện thế về trạng thái nghỉ
        \item Tạo giai đoạn refractory, ngăn phát xung liên tục
    \end{itemize}
\end{itemize}

% ============================================================
\section{Task 2: Leaky Integrate-and-Fire Model}

\subsection{Yêu cầu}

Mô phỏng điện thế màng của neuron khi nhận dòng điện đầu vào dạng sóng vuông (square wave).

\subsection{Phương trình mô hình}

\begin{equation}
\tau_m \frac{dV}{dt} = -(V - E_L) + R_m I_{\text{ext}}
\end{equation}

trong đó $\tau_m = C_m / g_L$ là hằng số thời gian màng.

\textbf{Cơ chế phát xung:}
\begin{itemize}
    \item Nếu $V \geq V_{\text{threshold}}$: phát xung (spike)
    \item Sau đó: $V \leftarrow V_{\text{reset}}$
    \item Neuron vào giai đoạn trơ (refractory period) trong thời gian $t_{\text{ref}}$
\end{itemize}

\subsection{Tham số mô hình}

\begin{table}[H]
\centering
\caption{Tham số LIF và dòng điện đầu vào}
\begin{tabular}{lcc|lcc}
\toprule
\multicolumn{3}{c|}{\textbf{Tham số neuron}} & \multicolumn{3}{c}{\textbf{Tham số sóng vuông}} \\
\midrule
$C_m$ & 1.0 & $\mu$F/cm$^2$ & Biên độ & 2.0 & $\mu$A/cm$^2$ \\
$g_L$ & 0.1 & mS/cm$^2$ & Chu kỳ & 50.0 & ms \\
$E_L$ & -65.0 & mV & Duty cycle & 50\% & \\
$V_{\text{threshold}}$ & -50.0 & mV & Bắt đầu & 10.0 & ms \\
$V_{\text{reset}}$ & -70.0 & mV & & & \\
$t_{\text{ref}}$ & 2.0 & ms & & & \\
\bottomrule
\end{tabular}
\end{table}

\subsection{Kết quả mô phỏng}

\begin{figure}[H]
\centering
\includegraphics[width=0.95\textwidth]{lif_static_plots.png}
\caption{Mô phỏng LIF với sóng vuông: (a) Điện thế màng với 4 spikes được phát hiện, (b) Dòng điện đầu vào dạng sóng vuông với chu kỳ 50 ms.}
\end{figure}

\begin{figure}[H]
\centering
\includegraphics[width=0.75\textwidth]{/mnt/project/lif_animation.gif}
\caption{Animation quá trình tích hợp điện thế và phát xung của mô hình LIF.}
\end{figure}

\subsection{Phân tích kết quả}

\textbf{Thống kê phát xung:}
\begin{itemize}
    \item Tổng số xung: 4 spikes
    \item Xung đầu tiên: 23.80 ms
    \item Xung cuối cùng: 173.40 ms
    \item Tần số phát xung: 20.00 Hz (1 spike/chu kỳ)
\end{itemize}

\textbf{Quan sát:}
\begin{itemize}
    \item Khi $I_{\text{ext}} = 0$: $V$ ở trạng thái nghỉ $E_L = -65$ mV
    \item Khi $I_{\text{ext}} = 2.0$ $\mu$A/cm$^2$: $V$ tích hợp theo hàm mũ
    \item Khi $V$ đạt $V_{\text{threshold}} = -50$ mV: neuron phát xung
    \item Sau xung: $V$ reset về $-70$ mV và giữ nguyên trong 2 ms
    \item Neuron chỉ phát 1 xung trong mỗi chu kỳ ON của sóng vuông
\end{itemize}

% ============================================================
\section{Task 3: Reservoir Neural Network}

\subsection{Yêu cầu}

Sử dụng Reservoir Neural Network (Echo State Network) để dự đoán chuỗi thời gian Mackey-Glass, minh họa khả năng dự đoán chính xác cả giá trị tương lai gần ($x(t+10)$) và xa ($x(t+100)$).

\subsection{Chuỗi Mackey-Glass}

Phương trình vi phân trễ:
\begin{equation}
\frac{dx}{dt} = \frac{ax(t-\tau)}{1 + x(t-\tau)^{10}} - bx(t)
\end{equation}

\textbf{Tham số:} $a = 0.2$, $b = 0.1$, $\tau = 17$, $x_0 = 1.2$, $\Delta t = 0.1$

Với $\tau = 17$, hệ thống ở trạng thái hỗn loạn nhẹ (weakly chaotic).

\subsection{Kiến trúc Echo State Network}

\textbf{Phương trình ESN:}
\begin{align}
\mathbf{x}(t+1) &= \tanh(\mathbf{W}_{\text{in}}\mathbf{u}(t) + \mathbf{W}_{\text{res}}\mathbf{x}(t)) \\
\mathbf{y}(t) &= \mathbf{W}_{\text{out}}\mathbf{x}(t)
\end{align}

\textbf{Tham số mạng:}
\begin{itemize}
    \item $N_{\text{reservoir}} = 500$ neurons
    \item Spectral radius: $\rho(\mathbf{W}_{\text{res}}) = 0.9$
    \item Sparsity: 10\% kết nối
    \item Ridge regression: $\alpha = 10^{-6}$
\end{itemize}

\subsection{Kết quả dự đoán}

\begin{figure}[H]
\centering
\includegraphics[width=\textwidth]{reservoir_predictions.png}
\caption{Kết quả dự đoán chuỗi Mackey-Glass: (a)(b) Near-future prediction ($t+10$), (c)(d) Far-future prediction ($t+100$). Đường xanh: giá trị thực, đường đỏ: giá trị dự đoán.}
\end{figure}

\begin{figure}[H]
\centering
\includegraphics[width=0.75\textwidth]{/mnt/project/reservoir_animation.gif}
\caption{Animation quá trình dự đoán $t+10$ theo thời gian thực.}
\end{figure}

\subsection{Phân tích độ chính xác}

\begin{table}[H]
\centering
\caption{Đánh giá độ chính xác dự đoán}
\begin{tabular}{lccc}
\toprule
\textbf{Khoảng dự đoán} & \textbf{MSE} & \textbf{MAE} & \textbf{RMSE} \\
\midrule
Near-future ($t+10$) & 0.0026 & 0.0383 & 0.0512 \\
Far-future ($t+100$) & 0.1582 & 0.3419 & 0.3978 \\
\bottomrule
\end{tabular}
\end{table}

\textbf{Nhận xét:}

\begin{itemize}
    \item \textbf{Dự đoán $t+10$:} Rất chính xác (MSE = 0.0026)
    \begin{itemize}
        \item Đường dự đoán (đỏ) gần như trùng với giá trị thực (xanh)
        \item Reservoir học được động lực học ngắn hạn của hệ
    \end{itemize}
    
    \item \textbf{Dự đoán $t+100$:} Độ chính xác giảm đáng kể (MSE = 0.1582)
    \begin{itemize}
        \item Vẫn bắt được xu hướng tổng thể
        \item Hiệu ứng hỗn loạn làm giảm khả năng dự đoán xa
        \item Sai số tích lũy theo thời gian
    \end{itemize}
\end{itemize}

\subsection{Ưu điểm của Reservoir Computing}

\begin{enumerate}
    \item \textbf{Huấn luyện nhanh:} Chỉ huấn luyện lớp đầu ra bằng Ridge regression
    \item \textbf{Tránh vanishing gradient:} Không cần backpropagation through time
    \item \textbf{Bộ nhớ động:} Reservoir giữ thông tin quá khứ thông qua kết nối hồi quy
    \item \textbf{Khả năng xấp xỉ tổng quát:} Với reservoir đủ lớn, có thể xấp xỉ bất kỳ hệ động nào
\end{enumerate}

% ============================================================
\section{Kết luận}

\subsection{Tóm tắt}

Đề tài đã hoàn thành đầy đủ 3 nhiệm vụ mô phỏng:

\begin{enumerate}
    \item \textbf{Hodgkin-Huxley:} Mô phỏng chính xác action potential, phân tích rõ vai trò của $I_{\text{Na}}$ (depolarization) và $I_{\text{K}}$ (repolarization). Kết quả: $V_{\text{peak}} = 41.29$ mV, tần số phát xung ~10 Hz.
    
    \item \textbf{LIF:} Tái tạo thành công cơ chế phát xung khi nhận sóng vuông. Phát hiện 4 spikes với tần số 20 Hz, minh họa các đặc tính: threshold, reset, refractory period.
    
    \item \textbf{Reservoir Network:} Dự đoán chính xác chuỗi Mackey-Glass ở khoảng gần (MSE = 0.0026 cho $t+10$) và bắt được xu hướng ở khoảng xa (MSE = 0.1582 cho $t+100$).
\end{enumerate}

\subsection{Ý nghĩa}

\begin{itemize}
    \item Hiểu sâu về cơ chế hoạt động của neuron sinh học (HH model)
    \item Nắm vững phương pháp đơn giản hóa cho ứng dụng thực tế (LIF model)
    \item Tiếp cận công nghệ reservoir computing - một hướng đi mới trong deep learning
    \item Thành thạo kỹ năng mô phỏng, phân tích, và trực quan hóa dữ liệu khoa học
\end{itemize}

% ============================================================
\section*{Tài liệu tham khảo}
\addcontentsline{toc}{section}{Tài liệu tham khảo}

\begin{thebibliography}{9}

\bibitem{trappenberg2023}
Trappenberg, T. P. (2023). \textit{Fundamentals of computational neuroscience} (3rd ed.). Oxford University Press.

\bibitem{dayan2001}
Dayan, P., \& Abbott, L. F. (2001). \textit{Theoretical neuroscience: Computational and mathematical modeling of neural systems}. MIT Press.

\bibitem{tavanaei2019}
Tavanaei, A., et al. (2019). Deep learning in spiking neural networks. \textit{Neural Networks}, 111, 47--52.

\bibitem{lukosevicius2009}
Lukoševičius, M., \& Jaeger, H. (2009). Reservoir computing approaches to recurrent neural network training. \textit{Computer Science Review}, 3(3), 5--17.

\bibitem{hodgkin1952}
Hodgkin, A. L., \& Huxley, A. F. (1952). A quantitative description of membrane current and its application to conduction and excitation in nerve. \textit{The Journal of Physiology}, 117(4), 500--544.

\end{thebibliography}

\end{document}